\documentclass{article}
\usepackage{ctex}
\usepackage{makecell}
\usepackage{geometry}
\usepackage{multirow}
\usepackage{multicol}
\usepackage{fancyhdr}

\geometry{a4paper,left=25mm,right=20mm,top=25mm,bottom=25mm}

\title{暑期实训项目进度计划表}

\begin{document}

% \maketitle

\begin{center}
    {\LARGE 暑期实训项目进度计划表}
\end{center}

% Table generated by Excel2LaTeX from sheet 'Sheet2'
\begin{table}[htbp]
    \centering
    \begin{tabular}{ccc}
        09021227 金\phantom{金}桥 & 09021218 马晓龙 & 09021134 王智东 \\
        09021130 张博雅           & 09021202 钟世贵 & 09021215 曹江宁 \\
    \end{tabular}%
\end{table}%


\begin{table}[htbp]
    \centering
    \begin{tabular}{c|c|cc}
        \hline
        \multirow{4}[8]{*}{\makecell[c]{\textbf{第一周}}} & 进度                     & \multicolumn{2}{l}{正常}                                            \\
        \cline{2-4}                                       & \multirow{2}[4]{*}{状态} & \multicolumn{1}{c|}{客户端}      & \multicolumn{1}{l}{\makecell[l]{
        ·基本熟悉Compose框架的使用,可以根据文档编写界面                                                                                                   \\
        ·前端页面基本定型                                                                                                                                  \\
        ·客户端基本功能定型                                                                                                                                \\
        ·客户端与服务器端交互方式基本定型                                                                                                                  \\
        }}                                                                                                                                                 \\
        \cline{3-4}                                       &                          & \multicolumn{1}{c|}{服务器端}    & \multicolumn{1}{l}{\makecell[l]{
        ·完成服务器端技术选型                                                                                                                              \\
        ·基本确定服务器端业务逻辑                                                                                                                          \\
        ·完成了项目文件结构的搭建                                                                                                                          \\
        }}                                                                                                                                                 \\
        \cline{2-4}                                       & 计划                     & \multicolumn{2}{c}{\makecell[l]{
        ·在下周完成大部分前端页面                                                                                                                          \\
        ·在下周进行学籍管理功能对接联合调试                                                                                                                \\
        ·在下周完成大部分实体类以及服务类的编写                                                                                                            \\
        }}                                                                                                                                                 \\
        \hline
        \multirow{4}[8]{*}{\makecell[c]{\textbf{第二周}                                                                                                    \\ 前半周 }} & 进度                     & \multicolumn{2}{l}{正常}                                            \\
        \cline{2-4}                                       & \multirow{2}[4]{*}{状态} & \multicolumn{1}{c|}{客户端}      & \multicolumn{1}{l}{\makecell[l]{
        ·基本完成了GUI框架的搭建,实现了程序主体窗口以及导航栏等功能                                                                                       \\
        ·各个功能的子界面模板编写完成,后期可以方便调用                                                                                                    \\
        ·完成了登录界面的编写并实现登录逻辑,与服务器实现异步传输数据                                                                                      \\
        ·学籍管理功能的基本逻辑编写完成,与服务器端实现通信                                                                                                \\
        }}                                                                                                                                                 \\
        \cline{3-4}                                       &                          & \multicolumn{1}{c|}{服务器端}    & \multicolumn{1}{l}{\makecell[l]{
        ·完成了 Socket 通信基础框架的实现                                                                                                                  \\
        ·完成了服务端登录、鉴权与用户相关逻辑                                                                                                              \\
        ·完成了服务端数据库 DAO 的配置与实现                                                                                                               \\
        ·正在进行大部分实体类的编写                                                                                                                        \\
        ·正在进行大部分服务类的编写                                                                                                                        \\
        ·建立了统一的控制端逻辑,方便功能的增删                                                                                                            \\
        }}                                                                                                                                                 \\
        \cline{2-4}                                       & 计划                     & \multicolumn{2}{c}{\makecell[l]{
        ·在本周完成大部分数据库交互逻辑                                                                                                                    \\
        ·在本周完成大部分前端界面设计                                                                                                                      \\
        ·在下周进行前后端对接联合调试                                                                                                                      \\
        ·在下周完成基本功能的实现与测试                                                                                                                    \\
        }}                                                                                                                                                 \\
        \hline
    \end{tabular}%
\end{table}%


\end{document}