\documentclass{article}
\usepackage{ctex}
\usepackage{makecell}
\usepackage{geometry}
\usepackage{multirow}
\usepackage{multicol}
\usepackage{fancyhdr}
\usepackage{longtable}
\usepackage{multirow}
\usepackage{hyperref}
\usepackage{color}

\geometry{a4paper,left=25mm,right=20mm,top=25mm,bottom=25mm}

\title{vCampus项目总结}

\begin{document}

% \maketitle

\begin{center}
    {\LARGE vCampus项目总结}\\
    \vspace{0.5cm}
    \today\\
    组长邮箱:\href{mailto:jinqiao@seu.edu.cn}{\texttt{jinqiao@seu.edu.cn}}
\end{center}

\section{小组概况}

% Table generated by Excel2LaTeX from sheet 'Sheet2'
\begin{table}[htbp]
    \centering
    \setlength{\tabcolsep}{4mm}{
        \begin{tabular}{cccl}
            \hline
            \multicolumn{1}{c}{\textbf{组长}} & 09021227 & 金\phantom{金}桥 & 实现了项目的绝大多数前端页面                         \\
            \hline
            \multicolumn{1}{c}{\textbf{组员}} & 09021218 & 马晓龙           & 实现了项目的整体框架配置与一部分前端页面以及后端接口 \\
                                              & 09021134 & 王智东           & 实现了学籍模块的后端接口以及一部分前端页面           \\
                                              & 09021130 & 张博雅           & 实现了商店模块的后端接口                             \\
                                              & 09021202 & 钟世贵           & 实现了图书馆模块的后端接口以及一部分前端页面         \\
                                              & 09021215 & 曹江宁           & 实现了教务模块的后端接口                             \\
            \hline
        \end{tabular}}
\end{table}%

\section{前言}

本系统为东南大学计算机科学与技术专业2021级暑期学校专业技能实训课程内容。本文档总结了整个项目开发过程中的相关记录,以供后续查阅。

\section{项目需求}

本项目的目的是设计并使用Java实现一个虚拟校园系统,采用客户端/服务器端架构进行实现。

本项目的必做模块包括用户管理、学生学籍管理、选课系统、图书馆以及商店。需要对相应模块的具体功能进行实现。

\section{项目设计与开发}

项目设计上,我们采用了客户端/服务器端进行实现。通过在服务器端部署数据库,并且建立服务器端与客户端的通信。客户端负责页面的展示,服务器端负责数据库交互。

在项目的开发上,我们力求与真实的工业界开发接轨,切入真实业务场景而非简单敷衍了事。易用的界面、稳定可靠高性能的框架是我们开发时的准则。我们的关键技术均来自工业界主流解决方案。

\section{项目关键技术}

\subsection{Compose Desktop}

我们采用 Compose Desktop 作为前端 UI 的整体框架。它具有强大的跨平台可移植性,可以方便的移植到安卓、iOS等平台。除此以外强大的组件库也是我们选择它的一个主要原因。我们设计了大量的组件以满足项目的需求,实现了高程度的代码复用能力。

\subsection{Hibernate}

我们采用 Hibernate 作为与 MySQL 交互的 DAO 层。它是完全的面向对象编程、摒弃了传统的数据库中心的思想。此外,它实现的透明持久化也是我们选择的一个原因。无侵入性以及优良的移植性也是 Hibernate 作为业界主流解决方案的几个优势所在。

\subsection{Netty}

我们采用了 Netty 作为 Socket 的实现。稳定可靠且高性能是我们选择 Netty 的一个重要原因。作为目前工业界流行的 NIO 框架,它提供了 Socket 的可靠解决方案。

\subsection{Slf4j与Logback}

详细快速的日志系统无论是对于开发还是运维都是必不可少的。对此,我们采用了 Slf4j 与 Logback 作为日志系统。两者均为工业界流行的日志系统,为我们的项目提供了稳定可靠高性能的日志解决方案。

\section{项目经验及存在问题}

\subsection{经验}

经过为期四周的开发,我们一致认为,可靠的团队成员是整个项目最后得以完成的基础。除此以外,高效的工作效率亦离不开高效的工具,与工业界主流解决方案接轨是我们项目最后能够以较高完成度实现的重要条件。

\subsection{存在问题}

目前,团队的主要问题在于各个人的开发能力不相同,导致工作量上存在一定的差异。我们通过尽量均衡工作任务的方式弥补了这个问题。

\subsection{团队沟通和工作方式}

团队的沟通主要以线下辅以线上的方式实现。我们采用 Git 作为版本管理工具,整个项目托管于 GitHub 上并配置 CI/CD 自动化工作流。团队成员可以随时将自己的更改提交至仓库中并实时查看部署结果,实现了高效的工作效率。


\vfill
\noindent\textcolor{red}{教师点评}【优~~~~良~~~~中~~~~及格~~~~不及格】

\end{document}
